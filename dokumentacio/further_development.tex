\chapter{További fejlesztési lehetőségek}
A jelenlegi megvalósítás több irányban is továbbfejleszthető. Az alábbiakban a leginkább releváns, a felhasználói élményt és a funkcionalitást érdemben javító fejlesztési irányokat foglalom össze.

\paragraph{Vágányút előjegyzése és tárolása}
Jelenleg hiányzik a vágányutak előzetes beállításának és tárolásának lehetősége. A funkció lehetővé tenné vágányút előjegyzését akár foglalt vágányra is; amint a vágány felszabadul, a kapcsolódó jelző automatikusan szabadra (zöldre) állna. Ez csökkentené a kezelői beavatkozások számát, és javítaná a kezelhetőséget, mivel nem lenne szükség minden egyes vonatindítás előtt az előző vonat elhaladására várni.

\paragraph{A térképen kívüli forgalom kezelése}
A rendszer jelenleg nem támogatja a térképen kívülről érkező vagy oda távozó vonatok kezelését. Az ilyen vonatok legfeljebb tárolóvágányra helyezhetők, azonban integrált kezelésük hiányzik. A funkció bővítése lehetővé tenné a térképen kívüli vonatok megjelenítését, illetve a térképen belüli állomások közötti közlekedésük ütemezett irányítását.

\paragraph{Szimuláció kiértékelése és visszajelzés}
Indokolt a szimuláció eredményeinek kvantitatív kiértékelése, különös tekintettel a menetrendi pontosságra. Célszerű lenne a menetrend-szerkesztőben olyan visszajelző mechanizmus bevezetése, amely a két állomás közötti távolság és pályajellemzők alapján becsült menetidőt, valamint várható tartalékidőt jelez.

\paragraph{Hálózatos, többfelhasználós irányítás}
A többfelhasználós, hálózati üzemmód lehetővé tenné, hogy több kezelő egyidejűleg dolgozzon ugyanazon szimuláción. Ez nagyobb terület és bonyolultabb forgalmi helyzetek hatékony kezelését tenné lehetővé, akár a jelenlegi lefedettség többszörösén.