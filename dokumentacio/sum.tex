\chapter{Összefoglalás} \label{ch:summary}

Dolgozatomban a vasúti közlekedésfejlesztés egyik legaktuálisabb problémakörével, a menetrend-alapú infrastruktúra-tervezés szoftveres támogatásával foglalkoztam. A kiinduló premissza az volt, hogy a modern vasúti beruházások hatékonysága csak úgy növelhető, ha a tervezési folyamat során a szolgáltatási célok (menetrend) és a fizikai megvalósítás (pálya) nem szeparáltan, hanem integráltan jelennek meg. A hagyományos, tisztán építőmérnöki szemléletű kapacitásbővítés gyakran eredményezett túlméretezett vagy funkcionálisan nem megfelelő műszaki megoldásokat, ami alacsonyabb gazdasági megtérüléshez vezetett.

Ennek a problémának a feloldására terveztem és valósítottam meg egy olyan offline asztali alkalmazást, amely egyesíti a pályaszerkesztő és a menetrend-tervező szoftverek funkcionalitását. A fejlesztés során bemutattam, hogyan képezhető le szoftveresen a vasúti infrastruktúra topológiája és a rajta közlekedő járművek dinamikája úgy, hogy az a felhasználó számára ergonomikus és átlátható maradjon. A program architektúrája lehetővé teszi a viszonylatok, a menetidők és az állomási technológiák (pl. vágányfoglaltságok) rugalmas kezelését, támogatva ezzel az Integrált Ütemes Menetrend (ITF) szigorú követelményrendszerének való megfelelést.

A szoftver validációja és a tesztesetek futtatása során bebizonyosodott, hogy az alkalmazás nem pusztán egy adminisztratív nyilvántartó rendszer, hanem egy aktív döntéstámogató eszköz. A dolgozat egyik legfontosabb eredménye annak demonstrálása, hogy a program alkalmas az egyes vonalakon tervezett extra kapacitáshoz szükséges infrastrukturális beruházások szükségességének precíz tesztelésére. A szimulációs környezetben a tervező képes modellezni, hogy egy tervezett járatsűrítés vagy új viszonylat bevezetése megoldható-e a meglévő infrastruktúrán pusztán a menetrend finomhangolásával, vagy elkerülhetetlen a fizikai beavatkozás.

Az alkalmazás segítségével egzakt módon meghatározható az a „minimálisan szükséges” műszaki tartalom – például egy új kitérő beépítése, egy állomási vágány meghosszabbítása vagy egy rövidebb kétvágányú szakasz (repülővágány) kiépítése –, amely már elégséges a kívánt menetrendi stabilitás fenntartásához. Ezzel a módszerrel a beruházási költségek jelentősen optimalizálhatók, hiszen elkerülhetők a „biztonsági játékból” fakadó felesleges kapacitásbővítések.

Összegezve, a létrehozott szoftver sikeresen valósította meg a kitűzött célt: egy olyan eszközt ad a közlekedéstervezők kezébe, amelyben a menetrend és a pálya összhangja folyamatosan ellenőrizhető. A program használatával a „menetrend az első” elv a gyakorlatban is érvényesíthetővé válik, biztosítva ezzel, hogy a jövő vasúti fejlesztései ne csak műszakilag legyenek kivitelezhetőek, hanem forgalmilag indokoltak és gazdaságilag is megtérülők legyenek.