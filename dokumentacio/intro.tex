\section*{Absztrakt}
A vasúti közlekedés versenyképességének növelése, valamint a rendelkezésre álló erőforrások hatékony felhasználása napjainkban a közlekedéstervezés kiemelt kihívásai közé tartozik. Európában és a korszerű vasúti tervezési gyakorlatban paradigmaváltás figyelhető meg: a hagyományos, infrastruktúra-vezérelt megközelítést fokozatosan felváltja a menetrend-alapú tervezés (timetable-based planning). Ennek lényege, hogy az infrastruktúra-fejlesztés nem öncélú, hanem a kívánt szolgáltatási szint – jellemzően az Integrált Ütemes Menetrend (ITF) – megvalósítását szolgálja.

Jelen szakdolgozat célja egy olyan, offline módon használható asztali alkalmazás tervezése és implementálása, amely támogatja ezt a korszerű metodikát. A fejlesztett szoftver lehetővé teszi a vasúti pálya (infrastruktúra) és a menetrendi struktúra együttes, iteratív kezelését. A rendszerben a felhasználó párhuzamosan módosíthatja a pálya geometriai elemeit – például a kitérők elhelyezését, vágánykapcsolatok kialakítását vagy repülővágányok tervezését – és a viszonylatok menetrendjét. Az alkalmazás központi funkciója a két alrendszer közötti kompatibilitás vizsgálata: a szoftver visszajelzést ad arról, hogy a tervezett infrastruktúra képes-e a kívánt ütemes menetrend konfliktusmentes lebonyolítására.

A dolgozat bemutatja a szoftver architektúráját, az alkalmazott algoritmusokat, valamint a felhasználói felület ergonómiai kialakítását, amely elősegíti a komplex hálózati összefüggések átláthatóságát. Az eredmény egy olyan tervezéstámogató eszköz, amellyel minimalizálható a felesleges beruházások kockázata, és maximalizálható a gazdasági megtérülés azáltal, hogy csak a menetrendi stabilitáshoz feltétlenül szükséges beavatkozások kerülnek kijelölésre.

\section*{Rövidítésjegyzék}
\begin{description}
	\item[JSON] JavaScript Object Notation
	\item[ITF] Integrált Ütemes Menetrend
	\item[UI] User Interface (Felhasználói felület)
	\item[GUI] Graphical User Interface (Grafikus felhasználói felület)  
\end{description}

\section*{Fogalomjegyzék}
\begin{description}
	\item[Repülő kereszt] Egyvágányú pályán olyan pályaszakasz, ahol hosszabb szakaszon két vágány fut párhuzamosan, lehetővé téve, hogy a szembeközlekedő vonatok megállás nélkül kerüljék ki egymást.
	\item[Biztosítóberendezés] Olyan műszaki rendszer, amely a vasúti közlekedés biztonságát szolgálja, beleértve a jelzők, váltók és egyéb forgalomirányító eszközök működtetését és felügyeletét.
	\item[Jelző (szemafor)] Olyan vizuális eszköz, amely a vonatvezetők számára információt nyújt a továbbhaladás feltételeiről (például sebességkorlátozás).
	\item[Vágánykapcsolat] Olyan pályaszakasz, amely lehetővé teszi a vonatok számára, hogy egyik vágányról a másikra áthaladjanak, például kitérők vagy kereszteződések formájában.
	\item[Fedezés ("jelző fedez")] A vasúti jelzők által biztosított védelmi zóna, amely megakadályozza, hogy egy vonat egy adott szakaszra lépjen, amíg az előző vonat még azon a szakaszon tartózkodik.
	\item[Foglalt] Egy adott pályaszakasz vagy térköz azon állapota, amikor azon egy vonat tartózkodik, és így más vonatok számára nem elérhető.
	\item[Térköz] A vasúti pálya két jelző vagy állomás közötti szakasza, amelyen a vonatok közlekednek. Egy adott térköz egyidejűleg csak egy vonat számára lehet foglalt.
	\item[Automata térközjelző] Olyan jelzőrendszer, amely automatikusan kezeli a térközök foglaltságát és a vonatok közlekedését anélkül, hogy emberi beavatkozásra lenne szükség. A vonat elhaladása után a jelző automatikusan engedélyezi a következő vonat számára a továbbhaladást.
	\item[Vágányút] A vasúti pálya azon vágányszakaszainak összessége, amelyet egy adott jelző vagy biztosítóberendezés vezérel egy vonat számára a közlekedés során.
\end{description}


\chapter{Bevezetés}
\label{ch:intro}

\section{Tématerület ismertetése} A fenntartható mobilitás és a gazdasági hatékonyság követelménye alapjaiban formálja át a közlekedésfejlesztési projektek előkészítését. A nagyberuházások, különösen a kötött pályás infrastruktúrák esetében, magas költségigénnyel és hosszú élettartammal járnak, így a tervezési fázisban elkövetett hibák vagy szuboptimális döntések évtizedekre meghatározhatják egy régió közlekedésének minőségét és költséghatékonyságát. A dolgozat témája a vasúti infrastruktúra-beruházások tervezését támogató szoftveres megoldások vizsgálata és fejlesztése, különös tekintettel a menetrend és a pályaadottságok közötti szoros korrelációra. A fejlesztés a modern európai trendekhez illeszkedve a szolgáltatás-orientált megközelítést helyezi a középpontba, ahol a menetrend nem az infrastruktúra passzív eredménye, hanem a tervezés aktív alakítója.

\section{Motiváció és célkitűzés} A témaválasztást személyes és szakmai indíttatás egyaránt vezérelte. Gyermekkorom óta kiemelt figyelemmel kísérem a tömegközlekedési rendszerek működését, és az évek során szerzett tapasztalatok rávilágítottak a hazai és nemzetközi gyakorlatban előforduló tervezési anomáliákra. Számos esetben tapasztalható, hogy jelentős tőkeinjekcióval megvalósuló beruházások nem hozzák a várt szolgáltatási színvonal-emelkedést, vagy a ráfordított forrásokból lényegesen hatékonyabb rendszert lehetett volna létrehozni gondosabb, rendszerszemléletű tervezéssel.

Jó példa erre a Balatoni vasútvonal egyes szakaszainak kétvágányúsítása, vagy a budapesti 4-es metró megállókiosztása, ahol az infrastrukturális adottságok és a valós utasforgalmi/menetrendi igények közötti diszkrepancia figyelhető meg. Meggyőződésem, hogy a „hardver” (pálya) és a „szoftver” (menetrend) szétválasztott kezelése elavult; a jövő a két terület integrált szimulációjában rejlik.

A szakdolgozat elsődleges célja egy olyan szoftvereszköz létrehozása, amely áthidalja a szakadékot a pályaépítési és a menetrend-szerkesztési fázisok között. Célom egy olyan ergonomikus, könnyen kezelhető alkalmazás fejlesztése, amelyben a tervező mérnökök képesek a menetrendi koncepciókat (különösen az ütemes menetrendeket) közvetlenül rávetíteni a tervezett infrastruktúrára, azonnal detektálva a szűk keresztmetszeteket vagy a felesleges kapacitásokat.

\section{Háttér és a menetrend-alapú tervezés koncepciója} A nagy volumenű, kötött pályás infrastrukturális beruházások tervezése és kivitelezése a közlekedésfejlesztés legköltségesebb és legkomplexebb feladatai közé tartozik. A modern vasúti és elővárosi közlekedés fejlesztésében paradigmaváltás figyelhető meg: a hangsúly a puszta infrastruktúra-építésről áttevődik a szolgáltatás-orientált, úgynevezett menetrend-alapú tervezésre. A dolgozatban bemutatott alkalmazás célja e folyamat támogatása olyan iteratív szimulációs környezet biztosításával, amely lehetővé teszi a menetrendi koncepció és a pályageometria együttes optimalizálását.

A hagyományos megközelítésben gyakran először valósul meg az infrastruktúra fejlesztése, és a szolgáltatást (menetrendet) a már megépült fizikai korlátokhoz igazítják. Ezzel szemben a fejlesztett szoftver alapvetése, hogy a legjobb költség-haszon arányú ráfordításhoz folyamatos iteráció szükséges a tervezési fázisban. Az alkalmazás módot ad arra, hogy a felhasználó párhuzamosan módosítsa a menetrendi struktúrát és a szükséges infrastruktúra elemeit (például kitérők, jelzők elhelyezése, sebességkorlátozások), egészen addig, amíg a rendszer el nem éri a kívánt egyensúlyi állapotot.

Kiemelt tervezési szempont volt, hogy az alkalmazás specifikusan támogassa az ütemes menetrendek (ITF - Integrált Ütemes Menetrend) létrehozását és vizsgálatát. A hazai és nemzetközi tapasztalatok azt mutatják, hogy az utasok számára a kiszámítható, rendszeres időközönként ismétlődő, csatlakozásokra épülő menetrendi struktúrák a legvonzóbbak. A szoftver segítségével pontosan modellezhető, hogy egy adott ütemes struktúra bevezetéséhez milyen minimális, de elégséges infrastrukturális beavatkozások szükségesek.

Összességében a fejlesztett alkalmazás nem helyettesíti az infrastruktúra-beruházást – hiszen a kapacitási problémák fizikai beavatkozást igényelnek –, hanem célzottabbá és optimalizáltabbá teszi azt. A cél egy olyan eszköz biztosítása, amely támogatja a „menetrend az első” elvet, ugyanakkor teret enged a szükséges infrastrukturális korrekciók szimulációjának is.


\section{Hazai példák}
Az utóbbi években Magyarországon is egyre hangsúlyosabbá vált a menetrend-alapú, szolgáltatás-orientált vasútfejlesztési szemlélet. A Budapesti Agglomerációs Vasúti Stratégia (BAVS) előkészítése során a korábbi, elsősorban elemi infrastruktúra-listákra támaszkodó megközelítést felváltotta az a módszer, amely a kívánt kínálati szintből (sűrű, csatlakozás-orientált ütemes elővárosi menetrend) vezeti le a szükséges beavatkozásokat. A stratégia különösen nagy hangsúlyt fektet a menetrendhez való igazodásra, az ütemesség, az átszállási kapcsolatok és a csomóponti szinkronizáció elsődlegességére.

Jól szemlélteti az integrált tervezés hatását az Esztergom–Budapest vasútvonal példája. A vonal felújítását és az ütemes, kiszámítható menetrend bevezetését követően a teljes éves utasszám mintegy 22 millióról 58 millióra nőtt. A jelentős növekedés rávilágít arra, hogy a kapacitásnövelés, a megbízhatóság és a menetrendi struktúra összehangolt optimalizációja lényegesen nagyobb keresletgeneráló hatást eredményez, mint önmagában egy elszigetelt infrastruktúra-fejlesztés.

Ezek a hazai tapasztalatok megerősítik a dolgozatban bemutatott szoftver célját: a menetrendi koncepcióból kiinduló, iteratív infrastruktúra-tervezés támogatását. Az alkalmazás olyan tervezési környezetet biztosít, amelyben a lehetséges ütemes menetrend előre definiálja az infrastruktúra kritikus elemeit (például repülő keresztek, állomási vágánykapacitás, jelzőtávolságok), elkerülve ezzel a túl- vagy alulméretezést és csökkentve a felesleges beruházások kockázatát.

Szintén menetrend-vezérelt beruházási logikát tükröz a jelenleg folyó egyik legnagyobb hazai vasúti projekt, a Déli Körvasút fejlesztése. A Kelenföld–Ferencváros közötti szakaszon a háromvágányú pálya kialakításának indoka nem pusztán kapacitásnövelés önmagáért, hanem az elővárosi és távolsági vonatok ütemes, sűrített menetrendjének, valamint a tehervonati és kerülő irányú forgalom konfliktusmentes elválasztásának biztosítása. A tervezett kínálati szint – a sűrűbb áthaladási és csatlakozási ritmus, városi átkötő funkció és a csomóponti terhelés kiegyenlítése – olyan headway és tartalékidő követelményeket támaszt, amelyek két vágányon már nem lennének megbízhatóan fenntarthatók csúcsidőben. Ez jól mutatja, hogy a menetrendi igény (kínálat) explicit meghatározása előzi meg és indokolja az infrastruktúra-bővítést.

