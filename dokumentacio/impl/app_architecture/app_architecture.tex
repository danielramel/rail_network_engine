\section{Szoftverarchitektúra és rendszerfelépítés}

Az alkalmazás fejlesztése során moduláris, rétegelt architektúrát alakítottam ki, amely elősegíti a kód átláthatóságát és a későbbi bővíthetőséget. A projekt forráskódja a \texttt{src} könyvtárban található, amely a felelősségi körök (Separation of Concerns) alapján az alábbi fő rétegekre és komponensekre tagolódik.

\subsection{Belépési pontok és vezérlés}
Az alkalmazás indításáért és az életciklus-kezelésért felelős fájlok:
\begin{itemize}
    \item \texttt{main.py}: A program belépési pontja. Feladata az alkalmazás indítása és a \texttt{MenuManager} inicializálása.
    \item \texttt{menu\_manager.py}: Az alkalmazás központi állapotkezelője. Feladata az alkalmazás inicializálása, valamint a beérkező események (billentyűzet, egér) továbbítása a megfelelő vezérlő felé.
    \item \texttt{home\_page\_screen.py}: A kezdőképernyő megjelenítéséért és interakcióiért felelős osztály. Lehetővé teszi új projektek létrehozását, meglévők betöltését, valamint a legutóbbi projektek gyors elérését.
\end{itemize}

\subsection{Core (Mag) réteg}

A \texttt{core} réteg képezi az alkalmazás gerincét; ez a modul kapszulázza az üzleti logikát, a domain modelleket, a konfigurációs állományokat és az alapvető infrastrukturális szolgáltatásokat. A réteg feladata biztosítani a rendszer belső konzisztenciáját, függetlenül a felhasználói felület (UI) aktuális megvalósításától.

\subsubsection{Config modul}
A konfigurációs modul felelős az alkalmazás globális paramétereinek, állandóinak és erőforrás-hivatkozásainak központosított tárolásáért.

\begin{description}
    \item[\texttt{color.py}:] A színpaletta definícióit tartalmazza, biztosítva a konzisztens vizuális megjelenést az egész alkalmazásban.
    \item[\texttt{keyboard\_shortcuts.py}:] A beviteli vezérlés absztrakciója; itt kerülnek definiálásra a módváltásokhoz és funkciókhoz rendelt gyorsbillentyűk.
    \item[\texttt{paths.py}:] Az erőforrás-kezelés segédosztálya, amely dinamikusan kezeli az ikonok és adatfájlok relatív és abszolút elérési útvonalait.
    \item[\texttt{config.py}:] Az alkalmazás központi konfigurációs osztálya. Tartalmazza a szimulációs és megjelenítési paramétereket (pl. rácsméret, sebességhatárok, fizikai állandók). A fájl tartalma a \ref{lst:config_class} kódrészleten látható.
\end{description}

\begin{lstlisting}[language=Python, caption={A Config osztály részlete, amely a szimuláció változtatható paramétereit definiálja}, label={lst:config_class}, inputencoding=utf8, extendedchars=true]
class Config:
    MAPS_FOLDER = "maps"
    GRID_SIZE = 40          # Pixel
    BUTTON_SIZE = 50
    STATION_RECT_WIDTH = 6
    STATION_RECT_HEIGHT = 1
    FPS = 20
    
    MIN_TRACK_SPEED = 10
    MAX_TRACK_SPEED = 200
    TRACK_SPEED_INCREMENT = 10
    
    SHORT_SECTION_LENGTH = 50
    LONG_SECTION_LENGTH = 250
    
    TRAIN_CAR_LENGTH = 25
    TRAIN_CAR_GAP = 5
    MAX_TRAIN_CAR_COUNT = 16
    MIN_TRAIN_STOP_TIME = 30  # mp
    TRAIN_SAFETY_BUFFER = 0   # m
\end{lstlisting}

\subsubsection{Graphics modul}
\begin{itemize}
    \item \textbf{\texttt{icon\_loader.py}:} Az erőforrások (sprite-ok, textúrák) betöltéséért és gyorsítótárazásáért felelős.
    \item \textbf{\texttt{camera.py}:} A felhasználói nézet transzformációit (eltolás, nagyítás/kicsinyítés) végző osztály. Lehetővé teszi a virtuális tér és a képernyő koordinátarendszere közötti konverziót.
    \item \textbf{\texttt{graphics\_context.py}:} Egy \textit{Context Object}, amely összefogja a rajzoláshoz szükséges függőségeket (pl. a Pygame felületét és a kamera objektumot), és továbbítja azokat a kirajzolást végző entitásoknak.
\end{itemize}

\subsubsection{Models csomag}
A \texttt{models} csomag tartalmazza a rendszer összes domain entitását, logikailag elkülönített alcsomagokba szervezve.

\paragraph{Geometry (Geometria és Térinformatika)}
A térbeli reprezentációért és a gráf-alapú koordinátákért felelős osztályok.
\begin{itemize}
    \item \textbf{\texttt{node.py}:} A diszkrét rácspontokat reprezentáló osztály. A gráf csomópontjaként funkcionál, tárolja a koordinátákat és a magassági szintet (pl. alagút kezeléséhez).
    \item \textbf{\texttt{position.py}:} A folytonos (Euclideszi) tér egy pontját reprezentálja lebegőpontos koordinátákkal. Míg a \texttt{Node} a gráf csomópontjait jelöli, a \texttt{Position} a képpontok és a kurzor finommozgását írja le.
    \item \textbf{\texttt{edge.py}:} A gráf éleinek absztrakciója, amely két \texttt{Node} közötti kapcsolatot definiál.
    \item \textbf{\texttt{direction.py}:} A vektorirányok kezelését végző segédosztály. Kulcsszerepe van a gráfbejárás során a lehetséges haladási irányok meghatározásában.
    \item \textbf{\texttt{pose.py}:} A \textit{Position} és \textit{Direction} kombinációja (pozíció és orientáció). Nélkülözhetetlen a jelzők és vonatok állapotának leírásához, ahol a térbeli elhelyezkedés mellett az irány is meghatározó.
\end{itemize}

\begin{lstlisting}[language=Python, caption={A Pose osztály \texttt{get\_connecting\_poses} metódusa, amely azokat a pozíciókat adja vissza, amelyek a vonatok által bevehető pályageometriájú szakaszokat reprezentálják}, label={lst:pose_get_connecting_poses}, inputencoding=utf8, extendedchars=true]
    def get_connecting_poses(self, other_level: bool = False) -> list['Pose']:
        neighbors = []
        for dir in self.direction.get_valid_turns():
            nx = self.node.x + dir.x
            ny = self.node.y + dir.y
            new_state = Pose(Node(nx, ny, self.node.level), dir)

            neighbors.append(new_state)
            if other_level:
                neighbors.append(new_state.toggle_level())
        return neighbors
\end{lstlisting}

\paragraph{Railway (Vasúti Logika)}
A különböző vasúti entitások és algoritmusait összekötő réteg.
\begin{itemize}
    \item \textbf{\texttt{railway\_system.py}:} Egy \textbf{Facade (Homlokzat)} tervezési mintát megvalósító osztály. Ez az egyetlen belépési pont a külvilág számára a vasúti alrendszer felé; összefogja és koordinálja az állomások, vonatok, jelzők és a biztosítóberendezés működését.
    \item \textbf{\texttt{graph\_adapter.py}:} Egy \textbf{Adapter} mintát követő osztály, amely elszigeteli a \textit{NetworkX} könyvtárat a rendszer többi részétől. Ez biztosítja, hogy a gráf-implementáció cseréje esetén csak ezt az osztályt kelljen módosítani.
    \item \textbf{\texttt{graph\_service.py}:} Magas szintű gráfműveleteket és a gráfon végzett keresési algoritmusokat (pl. szekciókeresés, peron-validáció) biztosító szolgáltatás.
    \item \textbf{\texttt{signalling\_service.py}:} A biztosítóberendezés logikáját implementálja. Feladata a jelzők aspektusának (szabad/tilos) dinamikus frissítése és a vágányutak foglaltságának ellenőrzése.
    \item \textbf{\texttt{path\_finder.py}:} A nem gráfon végzett útvonalkeresést megvalósító osztály. A \texttt{A*} algoritmust alkalmazza két pont közötti legrövidebb vágány építésének meghatározására.
\end{itemize}
\paragraph{Repositories (Adattárolók)}
A \textbf{Repository} tervezési minta alkalmazása az entitások életciklusának kezelésére. Ezek az osztályok felelősek az objektumok (vonatok, állomások, menetrendek) memóriában történő tárolásáért, lekérdezéséért és módosításáért.
\begin{itemize}
    \item \texttt{station\_repository.py}, \texttt{signal\_repository.py}, \texttt{train\_repository.py}, \texttt{timetable\_repository.py}
\end{itemize}

\paragraph{Domain Entities (Entitások)}
A rendszer alapvető építőkövei:
\begin{itemize}
    \item \textbf{\texttt{rail.py}:} A fizikai vágány modellje, amely kiterjeszti az \texttt{Edge} osztályt olyan tulajdonságokkal, mint a sebességkorlát és a pályahossz.
    \item \textbf{\texttt{signal.py}:} A vasúti jelzőberendezés modellje. Tárolja a pozíciót (\texttt{Pose}), a következő jelző referenciáját és a fedezett vágányutat.
    \item \textbf{\texttt{train.py}:} A vonat modellje. Az osztály kezeli a jármű fizikai paramétereit (sebesség, gyorsulás, fékezés), követi a menetrendet és interakcióba lép a pályával.
    \item \textbf{\texttt{timetable.py} és \texttt{schedule.py}:} A \texttt{Timetable} a statikus útvonaltervet (megállók sorrendje, és időkülönbsége), míg a \texttt{Schedule} egy konkrét, időponthoz kötött járatot reprezentál, amely alapján a szimuláció elindítja a vonatot.
\end{itemize}


\subsection{Shared (Megosztott) réteg}
\label{subsec:shared-layer}

A \texttt{shared} réteg alapvető célja a kód újrafelhasználhatóságának biztosítása a különböző modulok között. Ez a réteg tartalmazza az alapvető UI osztályokat, a vezérlő logikákat (controllers), valamint a rajzolást és egyéb segédfunkciókat megvalósító könyvtárakat.

\subsubsection{UI Modellek és Alaposztályok}
\label{subsubsec:ui-models}

Az \texttt{ui/models} könyvtár definiálja az összes felhasználói felület (UI) komponenst és azok absztrakt ősosztályait.

\begin{description}
    \item[\texttt{ui\_component.py}] 
    Az összes UI komponens absztrakt ősosztálya (\textit{Abstract Base Class}). Ez az osztály definiálja a komponensek alapvető viselkedését és interfészét.
    
    \begin{lstlisting}[language=Python, caption={Az UIComponent absztrakt osztály}, label={lst:ui_component}, inputencoding=utf8, extendedchars=true]
from core.models.geometry.position import Position
from core.models.event import Event
from abc import ABC, abstractmethod

class UIComponent(ABC):
    def dispatch_event(self, event: Event) -> bool:
        if hasattr(self, 'handled_events') and event.type not in self.handled_events:
            return False
        
        return self.handle_event(event)
    
    def handle_event(self, event: Event) -> bool:
        """Process an event that has already been filtered by type. Return True if consumed."""
        return False

    @abstractmethod
    def render(self, screen_pos: Position) -> None:
        """Render the UI component."""
        pass
    
    @abstractmethod
    def contains(self, screen_pos: Position) -> bool:
        """Check if a position is within the component's area."""
        return False
    
    def tick(self) -> None:
        """Advance the component's state by one tick."""
        pass
    \end{lstlisting}

    \item[\texttt{rectangle\_ui\_component.py}] 
    Téglalap alakú UI elemeket valósít meg. Az inicializáció során megadott befoglaló téglalapot elmenti az osztály állapotába, és implementálja a \texttt{contains} függvényt a geometriai vizsgálathoz.

    \item[\texttt{full\_screen\_ui\_component.py}] 
    A teljes képernyőt elfoglaló UI komponens. A \texttt{contains} metódusa minden esetben \texttt{True} értéket ad vissza, biztosítva, hogy minden eseményt érzékeljen a képernyőn.

    \item[\texttt{shortcut\_ui\_component.py}] 
    Billentyűparancsok kezelésére szolgáló komponens. A konstruktorban megadott billentyűkombináció figyelését végzi, és lenyomás esetén meghívja a hozzárendelt visszahívó (callback) függvényt.

    \item[\texttt{clickable\_ui\_component.py}] 
    Kattintható UI komponensek osztálya. Kiszűri a "húzott egér" (\textit{drag}) eseményeket, és valid kattintás esetén meghívja az \texttt{on\_click} metódust. A származtatott osztályoknak így csupán az \texttt{on\_click} logikát kell implementálniuk.

    \item[\texttt{button\_component.py}] 
    Összetett komponens, amely egyesíti a \texttt{shortcut} és a \texttt{clickable} osztályok funkcionalitását.

    \item[\texttt{panel.py}] 
    A \texttt{rectangle} és \texttt{clickable} osztályok leszármazottja. Alapértelmezetten definiál egy panelt a képernyő alsó középső részén. A konstruktor paraméterei lehetővé teszik a pozíció és méret testreszabását, valamint segédfunkciókat biztosít a betűtípusok kezeléséhez. A \texttt{render} metódus felelős a háttér és a keret kirajzolásáért, így a származtatott osztályok kizárólag a tartalom megjelenítésére koncentrálhatnak.

    \item[\texttt{ui\_controller.py}] 
    Az UI komponensek központi gyűjtője és kezelője. 
    \begin{itemize}
        \item \textbf{Eseménykezelés:} Az eseményeket a komponensek definiált sorrendjében továbbítja. Amennyiben egy komponens lekezeli az eseményt, a továbbítás megáll (chain of responsibility minta).
        \item \textbf{Kurzorkezelés:} Érzékeli, melyik komponens felett tartózkodik a kurzor. A takart komponensek felé nem továbbítja a kurzor pozícióját, így azok nem rajzolnak feleslegesen előnézetet (\textit{preview}).
        \item \textbf{Kirajzolás:} Felelős az összes regisztrált komponens \texttt{render} és \texttt{tick} metódusainak hívásáért.
    \end{itemize}
    \begin{lstlisting}[language=Python, caption={Az UIController osztály}, label={lst:ui_controller}, inputencoding=utf8, extendedchars=true]
from shared.ui.models.ui_component import UIComponent
from core.models.event import Event

class UIController(UIComponent):
    elements: tuple[UIComponent]
    
    def dispatch_event(self, event: Event):
        for element in self.elements:
            if element.dispatch_event(event):
                return True
            
        return False
    
    def render(self, screen_pos):
        elements_above_cursor = []
        if screen_pos is not None:
            for element in self.elements:
                elements_above_cursor.append(element)
                if element.contains(screen_pos):
                    break

        for element in reversed(self.elements):
            if element in elements_above_cursor:
                element.render(screen_pos)
            else:
                element.render(None)
                
    def tick(self):
        for element in self.elements:
            element.tick()
            
    def contains(self, screen_pos):
        return any(element.contains(screen_pos) for element in self.elements)
    \end{lstlisting}

\end{description}

A UI modellek közötti öröklődési és kapcsolati viszonyokat az \ref{fig:ui-models-uml}. ábra szemlélteti.

\begin{figure}[H]
    \centering
    \includegraphics[width=0.8\textwidth]{impl/app_architecture/ui_models_uml.png}
    \caption{Az UI komponensek UML osztálydiagramja}
    \label{fig:ui-models-uml}
\end{figure}

\subsubsection{Modellek (Models)}
\label{subsubsec:models}
\begin{description}
    \item[app\_state.py] Az alkalmazás állapotát reprenzentáló osztály. Tárolja az aktuális projektet, a mentési állapotot, valamint az alkalmazás fázisát (pl. \textit{Setup}, \textit{Simulation}).
\end{description}


\subsubsection{Vezérlők (Controllers)}
\label{subsubsec:controllers}

A \texttt{controllers} könyvtár tartalmazza az alkalmazás logikai vezérlőit.

\begin{description}
    \item[\texttt{app\_controller.py} (szülő: \texttt{ui\_controller})] 
    A projekt nézet belépési pontja. Feladatai közé tartozik:
    \begin{itemize}
        \item A projekt inicializálása vagy betöltése.
        \item A beérkező nyers események átalakítása az \texttt{event.py}-ban definiált belső eseménytípusokká.
        \item Hibák vizuális kezelése.
        \item Események továbbítása az UI komponensek felé.
    \end{itemize}

    \item[\texttt{app\_phase\_strategy.py} (szülő: \texttt{full\_screen\_ui\_component})] 
    Stratégia osztály, amely a projekt különböző fázisainak (pl. \textit{Setup}, \textit{Simulation}) állapotát kezeli és továbbítja a megfelelő modulok felé.

    \item[\texttt{camera\_controller.py}] 
    A nézet transzformációjáért felelős osztály. Kezeli a kamera pozícióját (pan) és nagyítási szintjét (zoom), valamint feldolgozza a vezérléshez kapcsolódó felhasználói bemeneteket.
\end{description}

\subsubsection{Megosztott UI Komponensek}
\label{subsubsec:shared-components}

A \texttt{components} könyvtárban találhatóak azok az általános elemek, amelyeket több projekt modul is felhasznál:

\begin{itemize}
    \item \textbf{alert\_component.py}: Általános figyelmeztető üzenetek és modális ablakok megjelenítésére szolgáló komponens.
    \item \textbf{input\_component.py}: Általános szövegbeviteli mező.
    \item \textbf{zoom\_button.py}: A nézet alaphelyzetbe állításáért és a jelenlegi nagyítás megjelenítéséért felelős gomb.
\end{itemize}

\subsubsection{Segédfunkciók és Szolgáltatások}
\label{subsubsec:utils-services}

A rendszer működését támogató egyéb osztályok és függvények.

\paragraph{Utils (Rajzolási segédfunkciók)}
A \texttt{utils} könyvtár tartalmazza a grafikus megjelenítéshez szükséges alacsony szintű függvényeket:
\begin{itemize}
    \item \textbf{grid.py}: A háttérrács kirajzolása.
    \item \textbf{lines.py}: Vonalak rajzolása, beleértve a szaggatott (\texttt{draw\_dashed\_line}) és pontozott (\texttt{draw\_dotted\_line}) stílusokat.
    \item \textbf{nodes.py}: Csomópontok (\texttt{draw\_node}) és elágazások (\texttt{draw\_junction\_node}) megjelenítése.
    \item \textbf{signal.py}: Vasúti jelzők rajzolása (\texttt{draw\_signal}).
    \item \textbf{draw\_station.py}: Állomások vizuális megjelenítése(\texttt{draw\_station}).
    \item \textbf{tracks.py}: Vágányok kirajzolása (\texttt{draw\_track}).
\end{itemize}

\paragraph{Services és Enums}
\begin{description}
    \item[\texttt{services/color\_from\_speed.py}] 
    Segédfunkció, amely a vágány sebessége alapján meghatározza annak megjelenítési színét, vizuális visszajelzést adva a sebességkorlátozásokról.
    \item[\texttt{enums/edge\_action.py}] 
    Enumeráció, amely a vágányokkal kapcsolatos kirajzolási és szerkesztési módokat definiálja.
\end{description}
\subsection{Setup Modul}
\label{subsubsec:setup-module}

Az alkalmazás alapértelmezett állapota; ez a modul töltődik be új projekt létrehozásakor vagy egy meglévő betöltésekor. A Setup modul két almodult foglal magában:
\begin{itemize}
    \item \textbf{Construction modul:} Pályaszerkesztési funkciók.
    \item \textbf{Train Placement modul:} Vonatok elhelyezése a pályán.
\end{itemize}

Architektúrális szempontból ezek az almodulok szándékosan nem a globális \texttt{shared} rétegben kaptak helyet, mivel funkcionalitásuk kizárólag a Setup fázisra korlátozódik. Ezzel a megoldással csökkenthető a globális névterek szennyezése és a felesleges függőségek kialakulása.

\paragraph{Vezérlők (Controllers)}
A modul működéséért felelős vezérlő osztályok:

\begin{description}
    \item[\texttt{setup\_mode.py}] 
    A Setup modul fő vezérlője. Felelős a két szerkesztési mód (építés és elhelyezés) közös funkcióinak vezérléséért, valamint a szimuláció indítása előtti inicializálási lépésekért (pl. vonatok alaphelyzetbe állítása).
    
    \item[\texttt{setup\_state.py}] 
    Állapotkezelő osztály, amely nyilvántartja az éppen aktív szerkesztési módot.
    
    \item[\texttt{setup\_mode\_strategy.py}] 
    A Strategy tervezési mintát megvalósító osztály, amely a Construction és Train Placement módok közötti dinamikus váltást és az azokhoz tartozó specifikus logikák cseréjét kezeli.
\end{description}

\paragraph{UI Komponensek}
A \texttt{ui} könyvtár tartalmazza mindkét Setup almodul által közösen használt felületi elemeket:

\begin{itemize}
    \item \textbf{Módválasztó és Vezérlés:}
    \begin{itemize}
        \item \texttt{setup\_mode\_selector\_buttons.py}: A két szerkesztési mód közötti váltást lehetővé tevő gombsor.
        \item \texttt{start\_simulation\_button.py}: A szerkesztés lezárását és a szimuláció indítását kezdeményező gomb.
        \item \texttt{timetable\_button.py}: A menetrend-szerkesztő felület megnyitására szolgáló gomb.
    \end{itemize}
    
    \item \textbf{Fájl- és Rendszerműveletek:}
    \begin{itemize}
        \item \texttt{save\_button.py}: A projekt aktuális állapotának mentése.
        \item \texttt{open\_button.py}: Meglévő projektfájl betöltése.
        \item \texttt{exit\_button.py}: Kilépés a Setup módból vissza a főmenübe (\textit{Home Page}).
    \end{itemize}
\end{itemize}
\subsection{Construction (Pályaszerkesztő) Modul}
\label{subsec:construction-module}

A Construction modul a Setup fázison belül a pályaszerkesztési és infrastruktúra-építési funkciókat valósítja meg. A modul architektúrája szigorúan szétválasztja a megjelenítést, az állapotkezelést és a vezérlést.

\begin{description}
    \item[\texttt{construction\_mode.py}]
    A modul központi vezérlője. Feladata a szerkesztési nézet inicializálása, az eszközök példányosítása és a fő eseményhurok kezelése a szerkesztés alatt.
\end{description}

\subparagraph{Modellek és Absztrakciók (\texttt{models})}
A \texttt{models} könyvtár tartalmazza a szerkesztőeszközök működéséhez szükséges ősosztályokat és a közös állapotkezelőt:

\begin{itemize}
    \item \textbf{construction\_state.py}: A szerkesztési mód állapotkezelője. Ez az osztály tárolja az aktuálisan kiválasztott eszközt, valamint a szerkesztési előnézetek (\textit{previews}) állapotát (pl. egy platform elhelyezésének vizualizációját).
    \item \textbf{construction\_tool\_panel.py}: Az egyes eszközökhöz (\textit{Tools}) tartozó információs és beállító panelek absztrakt ősosztálya.
    \item \textbf{construction\_tool\_controller.py}: Az egyes szerkesztőeszközök vezérlőinek közös ősosztálya.
    \item \textbf{construction\_tool\_view.py}: Az egyes eszközök nézeti logikájának absztrakt ősosztálya.
\end{itemize}

\subparagraph{Felhasználói Felület (\texttt{ui})}
A felhasználói felület elemei a közös keretrendszerre és a specifikus eszköz-implementációkra tagolódnak.

\begin{description}
    \item[\texttt{construction\_buttons.py}] A különböző szerkesztőeszközök közötti váltást biztosító gombsor.
    \item[\texttt{construction\_common\_view.py}] A szerkesztőnézet alaprétege. Felelős a statikus elemek (koordináta-rendszer, meglévő vágányok, jelzők, állomások, peronok) kirajzolásáért. Az aktív eszközök erre a rétegre rajzolják rá a saját dinamikus előnézeteiket, emellett bizonyos előnézeti logikákat is ez az osztály kezel.
    \item[\texttt{construction\_panel\_strategy.py}] A \textit{Strategy} tervezési minta alapján kezeli a dinamikus panelváltást, biztosítva, hogy mindig a kiválasztott eszközhöz (pl. váltóépítés) tartozó beállítások jelenjenek meg.
    \item[\texttt{construction\_tool\_strategy.py}] A \textit{Strategy} tervezési minta alapján kezeli az eszközök közötti váltást, biztosítva, hogy mindig a kiválasztott eszköz vezérlője és nézete legyen aktív.
\end{description}

\textbf{Szerkesztőeszközök (\textit{Tools}) felépítése:}
A rendszerben minden szerkesztőeszköz (pl. vágányépítő, jelzőlerakó) egységes, négy komponensből álló architektúrát követ:
\begin{enumerate}
    \item \textbf{Controller:} Kezeli a felhasználói interakciókat (kattintás, billentyűzet) és frissíti az eszköz belső állapotát.
    \item \textbf{View:} Megvalósítja a vizuális logikát, kirajzolja az eszköz-specifikus előnézeteket, és manipulálja a \texttt{construction\_state} előnézeti állapotait.
    \item \textbf{Panel:} Az eszközhöz tartozó specifikus beállításokat és a felhasználói instrukciókat megjelenítő felületi elem.
    \item \textbf{Target:} A nézet és a vezérlő közös geometriai logikáját tömörítő segédosztály. Feladata, hogy a kurzor pozíciója alapján azonosítsa a pályán végezhető lehetséges akciókat (pl. vágányvéghez való csatlakozás detektálása).
\end{enumerate}

A Construction modul és a benne található eszközök strukturális felépítését az \ref{fig:construction-ui-models-uml}. ábra szemlélteti.

\begin{figure}[H]
    \centering
    \includegraphics[width=0.8\textwidth]{impl/app_architecture/construction_module_uml.png}
    \caption{A Construction modul és az eszközök UML osztálydiagramja}
    \label{fig:construction-ui-models-uml}
\end{figure}
\subsection{Train Placement (Vonatelhelyező) Modul}
\label{subsec:train-placement-module}

A Train Placement modul a Setup fázison belül a vonatelhelyezési funkciókat valósítja meg. A modul architektúrája hasonló a construction modul-éhoz. A megjelenítés, az állapotkezelés és a vezérlés itt is szétválasztásra került.

\begin{description}
    \item[\texttt{train\_placement\_mode.py}]
    A modul központi vezérlője. Feladata a vonatelhelyezési nézet inicializálása, az eszközök példányosítása és a fő eseményhurok kezelése a vonatelhelyezés alatt.
\end{description}


\subparagraph{Modellek és Absztrakciók (\texttt{models})}
A \texttt{models} könyvtár tartalmazza a vonatelhelyező eszközök működéséhez szükséges ősosztályokat és a közös állapotkezelőt:
\begin{itemize}
    \item \textbf{train\_placement\_state.py}: A vonatelhelyezési mód állapotkezelője. Ez az osztály tárolja az aktuálisan kiválasztott eszközt, valamint a vonatelhelyezési előnézetek (\textit{previews}) állapotát (pl. egy törlendő vonat azonosítóját).
    \item \textbf{train\_placement\_tool\_controller.py}: Az egyes vonatelhelyező eszközök vezérlőinek közös ősosztálya.
    \item \textbf{train\_placement\_tool\_view.py}: Az egyes eszközök nézeti logikájának absztrakt ősosztálya.
\end{itemize}

\subparagraph{Felhasználói Felület (\texttt{ui})}
A felhasználói felület elemei a közös keretrendszerre és a specifikus eszköz-implementációkra tagolódnak.

\begin{description}
    \item[\texttt{train\_placement\_buttons.py}] A különböző vonatelhelyező eszközök közötti váltást biztosító gombsor.
    \item[\texttt{train\_placement\_common\_view.py}] A vonatelhelyezési nézet alaprétege. Felelős a statikus elemek (koordináta-rendszer, meglévő vágányok, jelzők, állomások, peronok) kirajzolásáért. Az aktív eszközök erre a rétegre rajzolják rá a saját dinamikus előnézeteiket, emellett bizonyos előnézeti logikákat is ez az osztály kezel.
    \item[\texttt{train\_placement\_panel\_strategy.py}] A \textit{Strategy} tervezési minta alapján kezeli a dinamikus panelváltást, biztosítva, hogy mindig a kiválasztott eszközhöz (pl. vonat elhelyezése) tartozó beállítások jelenjenek meg.
    \item[\texttt{train\_placement\_tool\_strategy.py}] A \textit{Strategy} tervezési minta alapján kezeli az eszközök közötti váltást, biztosítva, hogy mindig a kiválasztott eszköz vezérlője és nézete legyen aktív.
\end{description}

\textbf{Vonatelhelyező eszközök (\textit{Tools}) felépítése:}
A rendszerben minden vonatelhelyező eszköz (pl. vonat elhelyezése, vonat törlése) egységes, három komponensből álló architektúrát követ:
\begin{enumerate}
    \item \textbf{Controller:} Kezeli a felhasználói interakciókat (kattintás, billentyűzet) és frissíti az eszköz belső állapotát.
    \item \textbf{View:} Megvalósítja a vizuális logikát, kirajzolja az eszköz-specifikus előnézeteket, és manipulálja a \texttt{train\_placement\_state} előnézeti állapotait.
    \item \textbf{Panel:} Az eszközhöz tartozó specifikus beállításokat és a felhasználói instrukciókat megjelenítő felületi elem.
\end{enumerate}
A Train Placement modul és a benne található eszközök strukturális felépítését az \ref{fig:train-placement-ui-models-uml}. ábra szemlélteti.

\begin{figure}[H]
    \centering
    \includegraphics[width=0.95\textwidth]{impl/app_architecture/train_placement_module_uml.png}
    \caption{A Train Placement modul és az eszközök UML osztálydiagramja}
    \label{fig:train-placement-ui-models-uml}
\end{figure}
\subsection{Simulation (Szimulációs) Modul}
\label{subsec:simulation-module}

A Simulation modul a rendszer végrehajtó egysége, amely a vonatok valós idejű irányítását, a biztosítóberendezési logika (jelzők, vágányutak) érvényesítését és a fizikai szimuláció futtatását végzi. A modul architektúrája következetesen alkalmazza a modell-nézet-vezérlő (MVC) mintát, szigorúan szétválasztva az állapotkezelést, a megjelenítést és a vezérlési logikát.

\begin{description}
    \item[\texttt{simulation\_mode.py}]
    A szimulációs környezet belépési pontja és fő vezérlője (\textit{Main Controller}). Feladata a futtatókörnyezet inicializálásáért, beleértve a biztosítóberendezés (interlocking) logikájának példányosítását, valamint a központi eseményhurok felügyeletét.
\end{description}

\paragraph{Modellek (\texttt{models})}
Az üzleti logikát és az adatszerkezeteket tároló réteg:
\begin{description}
    \item[\texttt{simulation\_state.py}] 
    A szimuláció központi állapotkezelő entitása. Feladata a szimulációs idő szinkronizációja, az aktívan kijelölt járművek nyilvántartása, valamint a vágányút-beállítások vizuális előnézetének (\textit{preview}) menedzselése.
\end{description}

\subsubsection{Felhasználói Felület (\texttt{ui})}
A \texttt{ui} könyvtár tartalmazza a szimuláció vezérlését és a vizuális visszajelzést biztosító grafikus komponenseket.

\paragraph{Általános UI elemek}
\begin{description}
    \item[\texttt{simulation\_controller.py}] 
    A modul interakciós logikáját megvalósító osztály. Közvetít a felhasználói bemenetek (egér, billentyűzet) és a szimulációs modell között, biztosítva az állapotváltozások propagálását.
    
    \item[\texttt{simulation\_view.py}] 
    A grafikus megjelenítésért felelős osztály. Feladata a dinamikus objektumok (vonatok), a statikus infrastruktúra (jelzők, vágányok) és az aktív vágányutak valós idejű renderelése a képernyőre.
    
    \item[\texttt{time\_control\_buttons.py}] 
    A szimulációs idő manipulálását lehetővé tevő vezérlőfelület, amely funkciókat biztosít a szimuláció gyorsítására, lassítására vagy szüneteltetésére.
    
    \item[\texttt{time\_display.py}] 
    A szimulált rendszeridő digitális kijelzője.
    
    \item[\texttt{end\_simulation\_button.py}] 
    A szimulációs folyamat terminálására és a szerkesztő (Setup) módba való visszatérésre szolgáló vezérlőelem.
\end{description}

\paragraph{Vonatvezérlő Panelek (\texttt{ui/panel})}
A járműirányítás funkcionalitásának komplexitása indokolta, hogy a vonatvezérléssel kapcsolatos komponensek egy dedikált \texttt{panel} csomagba kerüljenek.

\begin{description}
    \item[\texttt{train\_panel\_manager.py}] 
    A járműpanelek életciklus-kezelője. Dinamikusan menedzseli a felhasználó által kijelölt vonatokhoz tartozó vezérlőfelületek példányosítását, megjelenítését és frissítését.
    
    \item[\texttt{train\_panel.py}] 
    Járműspecifikus irányítópult. Interfészt biztosít a mozdonyvezérléshez (indítás, irányváltás), továbbá valós időben megjeleníti a vonat telemetriai adatait (pl. aktuális sebesség) és a hozzárendelt menetrend státuszát.
    
    \item[\texttt{schedule\_selector.py}] 
    Menetrend-kiválasztó komponens (Qt widget integrációval). Lehetővé teszi, hogy a felhasználó az adatbázisban rendelkezésre álló menetrendek közül egyet hozzárendeljen az adott szerelvényhez.
\end{description}

A Simulation modul felépítését és az osztályok közötti kapcsolatokat a \ref{fig:simulation-module-uml}. ábra szemlélteti.

\begin{figure}[H]
    \centering
    \includegraphics[width=0.8\textwidth]{impl/app_architecture/simulation_module_uml.png}
    \caption{A Simulation modul és komponenseinek UML osztálydiagramja}
    \label{fig:simulation-module-uml}
\end{figure}
\subsection{Timetable (Menetrend) Modul}
\label{subsec:timetable-module}

A \texttt{timetable} modul biztosítja a vonatmenetrendek teljes körű kezelését, beleértve azok grafikus megjelenítését és. Technológiai szempontból ez a modul elkülönül a projekt többi részétől: a grafikus felhasználói felület (GUI) a \texttt{PyQt6} keretrendszer natív widgetjeire épül, kihasználva annak fejlett ablakkezelési és eseményvezérlési képességeit a komplex adatbeviteli feladatokhoz.

\subsubsection{Ablakok és Dialógusok}

\begin{description}
    \item[\texttt{timetable\_window.py}] 
    Egy \texttt{QDialog} (PyQt6) alapú osztály, amely a menetrend-kezelés belépési pontjaként szolgál. Feladata a rendszerben tárolt menetrendek listázása, rendszerezése és a kezelési műveletek (létrehozás, törlés, módosítás) koordinálása.

    \item[\texttt{timetable\_editor\_dialog.py}] 
    Egy \texttt{QMainWindow} (PyQt6) alapú összetett szerkesztőfelület. Ez az osztály valósítja meg a menetrendek részletes szerkesztéséhez szükséges logikát, lehetővé téve az állomások, megállási idők és indulási feltételek precíz konfigurálását.
\end{description}

\subsubsection{Stílusdefiníciók (\texttt{stylesheets})}

A felületek vizuális stílusát az alábbi fájlok definiálják:

\begin{itemize}
    \item \texttt{timetable\_editor\_stylesheet.py}: A szerkesztőablak komponenseinek vizuális szabálykészlete.
    \item \texttt{timetable\_window\_stylesheet.py}: A menetrend-választó ablak stílusleírója.
\end{itemize}
\subsection{Assets (Erőforrások) mappa}
\label{subsec:assets_folder}

A \texttt{assets} könyvtár az alkalmazás által használt statikus erőforrásokat tartalmazza, elsősorban a grafikus felhasználói felülethez (GUI) kapcsolódó ikonokat és képeket. Az ikonok a \texttt{assets/icons/} útvonalon érhetők el; mindegyik különálló, PNG formátumú fájl. A fájlnevek a betöltött funkciót tükrözik, ami támogatja a következetes hivatkozást és a könnyű karbantarthatóságot. Az ikonok és képek elsődleges forrása a The Noun Project (https://thenounproject.com/).

\paragraph{Fő ikonok}
Az általános, alkalmazásszintű ikonok a \texttt{assets/icons/} gyökérben találhatók:
\begin{description}
    \item[\texttt{app.png}] Az alkalmazás fő ikonja.
    \item[\texttt{save.png}] Mentés művelet ikonja.
    \item[\texttt{saved.png}] Mentett állapot vizuális jelzése.
    \item[\texttt{unsaved.png}] Módosított (nem mentett) állapot jelzése.
    \item[\texttt{open.png}] Megnyitás művelet ikonja.
    \item[\texttt{exit.png}] Kilépés művelet ikonja.
    \item[\texttt{construction.png}] Pályaszerkesztő mód jelzése.
    \item[\texttt{train\_placement.png}] Vonatelhelyező mód jelzése.
    \item[\texttt{timetable.png}] Menetrend ablak/mód jelzése.
\end{description}

\paragraph{Alkönyvtárak és modulikonok}
Az ikonok moduláris szervezésben, témakörönkénti alkönyvtárakban is megtalálhatók:
\begin{description}
    \item[\texttt{assets/icons/construction/}] Pályaszerkesztő módhoz kapcsolódó ikonok:
        \begin{description}
            \item[\texttt{bulldoze.png}] Törlés eszköz ikonja.
            \item[\texttt{rail.png}] Vágány eszköz ikonja.
            \item[\texttt{signal.png}] Jelző eszköz ikonja.
            \item[\texttt{station.png}] Állomás eszköz ikonja.
            \item[\texttt{platform.png}] Peron eszköz ikonja.
            \item[\texttt{tunnel.png}] Alagút eszköz ikonja.
        \end{description}
    \item[\texttt{assets/icons/train\_placement/}] Vonatelhelyező mód ikonok:
        \begin{description}
            \item[\texttt{place\_train.png}] Vonat elhelyezése ikon.
            \item[\texttt{remove\_train.png}] Vonat eltávolítása ikon.
        \end{description}
    \item[\texttt{assets/icons/simulation/}] Szimulációs vezérlés ikonok:
        \begin{description}
            \item[\texttt{play.png}] Szimuláció indítása.
            \item[\texttt{pause.png}] Szimuláció szüneteltetése.
            \item[\texttt{fast.png}] Gyorsított lejátszás.
            \item[\texttt{super\_fast.png}] Nagy mértékű gyorsítás.
        \end{description}
\end{description}

Az itt felsorolt struktúra a felhasználói felület komponensei közötti konzisztens ikonhasználatot szolgálja, és megkönnyíti az erőforrások azonosítását, cseréjét és bővítését.